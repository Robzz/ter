Ce document s'inscrit dans le cadre d'un travail d'étude et de recherche, en Master~2 Informatique et Sciences de l'Image, à
l'université de Strasbourg. Il décrit le travail effectué au cours du semestre de printemps 2015-2016 sur le sujet proposé,
une méthode de visualisation rapide de modèles 3D à partir de textures projectives de profondeur.

Le travail a été réalisé par Robin Chavignat, étudiant en M1 ISI, et encadré par Rémi Allègre, membre de l'équipe Informatique
Géométrique et Graphique au sein du laboratoire iCube.
