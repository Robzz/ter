\section{Introduction}
Il existe plusieurs façons de représenter un objet en 3D. La représentation la plus commune en rendu 3D et la représentation
par maillage. Un maillage consiste en la donnée d'un ensemble de polygones délimitant la surface de l'objet. On s'intéressera dans ce TER aux maillages
triangulaires, c'est à dire les maillages ne contenant que des triangles. En effet, tout polygone de 4 sommets ou plus pouvant être décomposé en un
ensemble de triangles, il est donc facile de se ramener au cas étudié.\\

\section{Formulation du problème}
La représentation par maillage souffre de plusieurs défauts. Celui nous intéressant le plus dans le cadre de ce TER est que pour certains objets,
il est impossible d'obtenir une représentation exacte. C'est le cas lorsque la surface de l'objet ne peut être représentée par un ensemble de triangles.
Par exemple, il est impossible de représenter une sphère de façon exacte sous forme de maillage. Les maillages représentant de tels objets sont des
approximations.\\
En conséquence, plus on désire un maillage précis, plus celui-ci contiendra de triangles. Un maillage plus complexe a un coût
plus important en mémoire, et en temps lors du rendu. Au delà d'une certaine précision, la représentation par maillage n'est plus viable
en terme de performances. 

% todo : chiffres, images

Il apparaît donc nécessaire de disposer de techniques permettant de représenter les détails d'un objet plus
efficacement qu'en augmentant la précision de son maillage.
