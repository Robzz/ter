Nous avons vu que la méthode fonctionne, mais également qu'elle présente quelques défauts. Voici mes pistes d'amélioration.
Mon implémentation n'étant pas fonctionnelle, il ne s'agit que de pistes, n'ayant pas été explorées.

\section{\'Elimination des occlusions}
Une amélioration possible serait un traitement permettant de rechercher des points/surfaces non couvertes par l'ensemble de points de vue
et souffrant du phénomène d'occlusion. Il serait ensuite possible de compléter l'ensemble par de nouveaux points de vue assurant une
meilleure visibilité de ces surfaces.

En observant le rendu des occlusions, on constate que les pixels atteints ``font t\^ache''. Cela s'explique par la nature des occlusions~:
le pixel dessiné est le résultat de la reprojection d'une mauvaise approximation du point de l'objet correspondant, ce qui explique la discontinuité observée.
Une piste prometteuse pour détecter les pixels touchés serait donc de calculer le gradient de couleur de l'image et de rechercher des discontinuités importantes.

\section{Automatisation de l'aquisition des points de vue}
Une des améliorations de la méthode sur la technique VDTM est la simplification des prétraitements. En effet, il
suffit de générer un maillage simplifié et de procéder a l'aquisition des points de vue. Cependant, cette dernière
partie requiert l'intervention de l'utilisateur. On se pose donc la question de comment automatiser ce traitement.\\
Une méthode naïve consisterait à construire un paramètrage de l'espace décrivant un ensemble de points servant de centres de projection
pour la capture des points de vue. On peut par exemple orienter la caméra vers le point de la surface le plus proche. Un exemple trivial
d'un tel paramétrage serait la sphère englobante de l'objet.\\
Cette méthode demande cependant à être affinée, car le nombre de points de vue en résultant peut devenir grand, et un critère d'orientation
peut donner de mauvais résultats.

\section{Elimination des points de vue redondants}
L'automatisation de l'aquisition risque dans tous les cas de générer un grand nombre de points de vue. Un algorithme
permettant d'éliminer de l'ensemble le point de vue apportant le moins d'information permettrait ainsi de réduire la taille
de l'ensemble tant que la suppression de points de vue ne cause pas de grosse perte de précision. Une ébauche d'un tel algorithme
peut se présenter ainsi~:

On cherche à associer à chaque point de vue de l'ensemble un coût, quantifiant la perte d'information causée par la suppression du point de vue.
La première étape est de déterminer pour chaque point de vue $V_i$ les régions de l'espace dans lesquelles placer le centre de projection de la caméra
provoquera l'utilisation de $V_i$ par la méthode. Il s'agit de l'ensemble des points $P$ tels que \[\forall{}j,k,i\neq{}j,i\neq{}k, d(P,V_i) < d(P,V_j) \text{ ou } d(P,V_i) < d(P,V_k)\]
On peut ensuite pour chacune de ces régions déterminer quel point de vue serait sélectionné à la place de $V_i$ s'il était supprimé de l'ensemble.
Il reste ensuite à faire un rendu dans les 2 cas, puis à quantifier la perte d'information d'un cas à l'autre, par rapport au rendu de référence.
Il est envisageable de répéter ce procédé tant que la perte ne dépasse pas un certain seuil.
