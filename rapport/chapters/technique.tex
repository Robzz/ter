De nombreuses techniques ont été développées pour répondre à ce besoin. On peut par exemple citer le normal mapping,
ou le displacement mapping. Les techniques que nous allons étudier appartiennent à la catégorie des algorithmes de
rendu à partir d'images, ou image-based rendering.\\

\section{Vue d'ensemble}
La méthode étudiée dans le cadre de ce TER est basée sur la technique de View-Dependent Texture Mapping (VDTM).
Le VDTM a été décrit en premier par Debevec et al~\cite{Debevec96}. Il consiste, à partir d'un maillage
approximatif de l'objet et d'un ensemble de prises de vues, à déterminer quels polygones sont visibles à partir
de chaque point de vue. Pour un point P de l'objet et pour une position de caméra donnée, on peut ensuite déterminer si P
est visible grâce aux informations de visibilité calculées précédemment. Si P est visible, sa couleur est déterminée
par une combinaison des couleurs de P dans les points de vue ou P est visible. Un défaut notable du VDTM est
que l'interpolation de couleurs entre les points de vue à tendance à causer des artefacts.

La technique que nous étudions comble cette dernière faiblesse du VDTM, tout en utilisant un prétraitement bien plus simples.
\'Etant donnés un maillage simplifié et un ensemble de points de vue, pour une position donnée de la caméra, on détermine pour tout pixel
quel est le point de vue fournissant la meilleure vue parmi un sous-ensemble de points de vue proches. On extrait ensuite
du point de vue les informations de couleur et de profondeur nécessaires à l'écriture du fragment. Ceci conduit à une
reconstruction approchée de l'objet original au dessus du maillage approximatif.

\section{Données}
Dans le cadre de la technique que nous étudions, les données nécessaires au rendu sont un maillage simplifié, les points de vue,
consistant en la donnée de 3 images, une carte de profondeur (z-buffer), une carte de normales, et une carte de couleurs, ainsi
que la matrice de projection de la caméra ayant capturé le point de vue.\\
Nous ne rentrerons pas dans les détails de la génération du maillage simplifié. De nombreux algorithmes permettent d'accomplir
cette tâche, comme par exemple ceux décrits par \cite{Hoppe96} et \cite{Garland97}. Les logiciels Blender ou Meshlab permettent
tous deux de générer le maillage simplifié à partir du maillage original, en spécifiant le degré de simplification, e.g 1\% du nombre
de triangles original.\\
Le prétraitement se limite à l'aquisition des points de vue, et peut se faire manuellement en naviguant autour de l'objet et en s'assurant
que l'ensemble de la surface est couverte. Nous décrivons plus tard des pistes envisagées pour automatiser l'étape d'aquisition des points de vue.

\section{Reprojection}
La technique VDTM et celle étudiée dans le cadre de ce TER sont toutes les deux basées sur l'utilisation de textures
projectives de profondeur, et sur le principe de reprojection suivant~:\\
Etant donné une caméra $C$ et la carte de profondeur associée à l'objet vu de $C$.
Soient $P = (x_P, y_P, z_P)$ un point sur la surface de l'objet, et $(u_P, v_P)$ les coordonnées du pixel correspondant
à $P$ dans la caméra $C$. Alors, le point $P_C = (u_P, v_P, z_P)$ avec $z_P$ extrait de la carte de pronfondeur, correspond à $P$
dans l'espace de la caméra $C$. Connaissant $M_C$ la matrice de projection de la caméra $C$, alors on a $P = M_C^{-1} * P_C$.

\section{Algorithme}

