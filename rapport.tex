\documentclass{report}

\usepackage[utf8]{inputenc}
\usepackage{amsmath}
\usepackage[ampersand]{easylist}
\usepackage{fancyhdr}
\usepackage{graphicx}

\begin{document}
\begin{titlepage}
    \centering
    \vfill
    \vfill
    {\bfseries\Large
        TER --- Rapport intermédiaire\\
        Visualisation rapide de modèles 3D complexes à partir de textures projectives de profondeur\\
        Robin Chavignat\\
    }    
    \vfill
    \vfill
    \includegraphics[width=12cm]{rapport_img.png} % also works with logo.pdf
    \vfill
    \vfill
\end{titlepage}
\pagestyle{fancy}
\section*{Problématique}
En rendu 3D, la représentation classique des objets par maillage n'est pas toujours adaptée. Si
un objet est suffisamment détaillé, le maillage devient très gros et n'est plus une façon efficace de représenter un
objet.\\
Dans un domaine ou il est acceptable d'utiliser une approximation de l'objet, on peut utiliser de nombreuses techniques
permettant de reconstruire les détails d'un objet à partir d'un maillage simplifié: normal mapping, displacement mapping, etc\ldots\\
Nous étudions dans le cadre de ce TER une technique dérivée du View-Dependant Texture Mapping (VDTM). Le VDTM a été décrit en premier par Debevec et
al~\cite{Debevec98}, et consiste à reconstruire une vue d'un objet à partir d'un ensemble d'images de l'objet, appelées
``points de vue'', et d'un maillage simplifié de l'objet.\\
La technique étudiée est basée sur le VDTM, et a été introduite par Porquet et al~\cite{Porquet05}. Les points de vue
consistent en des cartes de couleur, de profondeur, et de normales. Cette méthode a principalement 2 avantages par
rapport au VDTM~: la simplicité du pré-traitement (il suffit d'acquérir les points de vue) et la suppression de certains
artefacts causés par le blending entre plusieurs points de vue.\\
La méthode est principalement implémentée au niveau du fragment shader. On utilise les 3 points de
vue les plus proches de la caméra et pour chaque fragment, et on détermine le meilleur point de vue. L'objet est ensuite
dessiné en utilisant le meilleur point de vue pour chaque fragment.\\
La méthode permet d'obtenir des rendus proches de l'objet original avec un maillage très simplifié ($~0.1\%$ du nombre
de faces initial). On observe cependant un manque de précision au niveau des ``bords'' de l'objet, et la méthode ne
tient pas parfaitement compte des occlusions.\\
L'implémentation est réalisée en C++ et GLSL.\\
\pagebreak
\section*{Travail effectué}
\begin{easylist}[itemize]
    & Etude de la technique VDTM
    & Etude de la technique proposée par Porquet et al~\cite{Porquet05}
    & Recherche de maillages suffisamment complexes sur lesquels appliquer la technique, et simplification
    & Application OpenGL simple (textures, scenegraph, shaders, transformations, rendu offscreen, etc\dots)
    & Import de maillages au format Wavefront OBJ
    & Etude d'algorithmes de simplification de maillages~\cite{Hoppe96}~\cite{Garland97}
    & Chargement et sauvegarde de textures
    & Sauvegarde de points de vue
\end{easylist}
\pagebreak
\section*{Planning prévisionnel}
\begin{easylist}[itemize]
    & Terminer l'implémentation de la technique décrite dans~\cite{Porquet05}
    && Chargement de points de vue
    && Shaders
    && Ombres
    & Tests et benchmarks de l'implémentation
    & Recherches afin de tester la couverture du maillage par les points de vue
    & Automatisation de la capture des points de vue
\end{easylist}
\pagebreak
\bibliographystyle{plain-fr}
\bibliography{rapport}
\end{document}
